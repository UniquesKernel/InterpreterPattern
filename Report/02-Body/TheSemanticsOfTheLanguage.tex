\section{The Semantics}
The language implemented in this report is a short and simple language
that contains only one kind of terminal expression and four operators
for doing mathematical operations $+$, $-$, $*$, and $/$.

\begin{itemize}
  \item Numbers
  \item $+$ \textbar{} $-$ \textbar{} $*$ \textbar{} $/$
\end{itemize}

\noindent The language follow three simple rules and eliminates ambiguity
in a simple way. 

\begin{enumerate}
  \item All terminal expressions evaluate to themselves.
  \item All expressions are evaluated left to right
  \item All operators are applied to the two preceding expressions i.e.\\
    $<$Expression$>$ $<$Expression$>$ $<Op>$
\end{enumerate}

These rules ensures clearity in the evaluation of all expressions unlike the
example shown in \autoref{fig:AST_Ambiguity}. According to the simple language
defined above, the previous example would not be written as $ 4 * 5 + 2 $
but instead be written as $ 4\ 5\ *\ 2\ + $. The expression can then be represented 
by its own AST as seen in \autoref{fig:Non_Ambigous_AST}

\begin{figure}[h]
  \centering
  \tikzstyle{ASTNode} = [draw, circle]
  \begin{tikzpicture}[node distance = 1cm, edge/.style={link}]
    \node[ASTNode] (root) {$+$};
    \node[ASTNode] (left) [below left = of root] {$*$} edge (root);
    \node[ASTNode] (right) [below right = of root] {$2$} edge (root);
    \node[ASTNode] (left-left) [below left = of left] {$4$} edge (left);
    \node[ASTNode] (left-right) [below right = of left] {$5$} edge (left);
  \end{tikzpicture}
  \caption{AST for simple language with no ambiguity}%
  \label{fig:Non_Ambigous_AST}
\end{figure}

Note, because of rule 2 listed above there is no worry about ambiguity. The 
expression above can therefore only evaluate to 22.

\newpage
