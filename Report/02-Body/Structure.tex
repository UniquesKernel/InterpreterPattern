\section{Structure of the Interpreter Pattern}

The structure of the interpreter pattern is every simple.
Its starts out defining an abstract expression class, from which
two other abstract classes, the terminal expression class and non-terminal
expression class inherites the interpret method.

\begin{figure}[h]
  \centering
  \begin{tikzpicture}

    \umlclass[type=abstract, x = -5]{Context}{}{}
    \umlclass[type=abstract]{Expression}
    {Interpret(Context context)}{}
   
    \umlclass[type=abstract, x = -3, y = -3]{Terminal Expression}
    {Interpret(Context context)}{}

    \umlclass[type=abstract, x = 3, y = -3]{Non-Terminal Expression}
    {Interpret(Context context)}{}

    \umluniassoc[thick]{Context}{Expression}
    \umlinherit[thick, geometry=-|]{Terminal Expression}{Expression}
    \umlinherit[thick, geometry=-|]{Non-Terminal Expression}{Expression}

  \end{tikzpicture}
  \caption{The basic structure of the interpreter pattern}%
  \label{fig:basic_interpreter_structure}
\end{figure}

\noindent Notice the Context class, which so far has not been mentioned. The simple
language implemented in this paper does not use variables, this is 
where the context would come in. It is used to define a relationship between
variables and values, so through this one could declare that the variable
$ x = 10 $. It can often take the form of a Map or Stack data structure, but
since it is not used in this implementation, it is mentioned only for the sake
of completion.
