\section{Structure of the Interpreter Pattern}

The structure of the interpreter pattern is every simple.
Its starts out defining an abstract expression class, from which
two other abstract classes, the terminal expression class and non-terminal
expression class inherites the interpret method.

\begin{figure}[h]
  \centering
  \begin{tikzpicture}

    \umlclass[type=abstract, x = -5]{Context}{}{}
    \umlclass[type=abstract]{Expression}
    {Interpret(Context context)}{}
   
    \umlclass[type=abstract, x = -3, y = -3]{Terminal Expression}
    {Interpret(Context context)}{}

    \umlclass[type=abstract, x = 3, y = -3]{Non-Terminal Expression}
    {Interpret(Context context)}{}

    \umluniassoc[thick]{Context}{Expression}
    \umlinherit[thick, geometry=-|]{Terminal Expression}{Expression}
    \umlinherit[thick, geometry=-|]{Non-Terminal Expression}{Expression}

  \end{tikzpicture}
  \caption{The basic structure of the interpreter pattern}%
  \label{fig:basic_interpreter_structure}
\end{figure}

\noindent Notice the Context class, which so far has not been mentioned. The simple
language implemented in this paper does not use variables, this is 
where the context would come in. It is used to define a relationship between
variables and values, so through this one could declare that the variable
$ x = 10 $. It can often take the form of a Map or Stack data structure, but
since it is not used in this implementation, it is mentioned only for the sake
of completion.

\newpage

\subsection{Structure of implementation}
In the specific implementation, presented in this paper, the structure is
expanded to include the implemented derivatives of the terminal and non-terminal
branches as seen in \autoref{fig:implmented_structure}

\begin{figure}[h]
  \centering
  \begin{tikzpicture}[]

    \umlclass[type=interface]{Expression}
    {Interpret()}{}
   
    \umlclass[type=abstract, x = -3, y = -3]{Terminal Expression}
    {Interpret()}{}

    \umlclass[type=abstract, x = 3, y = -3]{BinaryOperator Expression}
    {Interpret()}{}
    \umlclass[type=abstract, x = -4, y = -7]{Integer Expression}
    {Interpret()}{}
    \umlclass[type=abstract, x = 0, y = -7]{Plus Expression}
    {Interpret()}{}
    \umlclass[type=abstract, x = 6, y = -7]{Minus Expression}
    {Interpret()}{}
    \umlclass[type=abstract, x = 0, y = -10]{Div Expression}
    {Interpret()}{}
    \umlclass[type=abstract, x = 6, y = -10]{Multiplication Expression}
    {Interpret()}{}

    \umlimpl[thick, geometry=-|]{Terminal Expression}{Expression}
    \umlimpl[thick, geometry=-|]{Non-Terminal Expression}{Expression}
    \umlimpl[thick, geometry=-|]{Integer Expression}{Terminal Expression}
    \umlimpl[thick, geometry=-|]{Plus Expression}{BinaryOperator Expression}
    \umlimpl[thick, geometry=-|]{Minus Expression}{BinaryOperator Expression}
    \umlimpl[thick, geometry=-|]{Div Expression}{BinaryOperator Expression}
    \umlimpl[thick, geometry=-|]{Multiplication Expression}{BinaryOperator Expression}
  
  \end{tikzpicture}
  \caption{Stucture of the implementation}%
  \label{fig:implmented_structure}
\end{figure}

\noindent
As can be seen in the implentation structure the the Non-Terminal Expression is replaced by a Binary 
Operator Expression. In this case a practical solution, but if other operators existed, this might
not be the best way to implement the interpreter pattern as their might be many different non-terminal
expressions. Though the Terminal expression could have been removed as well it serves well to remember
that, maybe terminal expression, like variables could be implemented, by keeping the abstraction in place.

\newpage
